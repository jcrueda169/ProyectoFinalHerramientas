\documentclass{article}

\usepackage[utf8]{inputenc}
\usepackage{longtable}
\usepackage{authblk}
\usepackage{adjustbox}
\usepackage{natbib}


\title{LOS INDICES DEL MUNDO}
% autores
\renewcommand\Authand{, y }
\author[1]{\normalsize Juan Sebastián Cortazar}
\author[2]{\normalsize María Alejandra Restrepo}
\author[3]{\normalsize Juan Camilo Rueda}

\affil[1,2,3]{\small  Universidad de los Andes\\
\texttt{{js.cortazar533,ma.restrepot,jc.rueda169}@uniandes.edu.col}}
\affil[1]{\small Basico2}
\texttt{{i-de@cdsystems.com.co}}
\affil[3]{\small Construcotra Colpatria\\
\texttt{{juanc.rueda@constructoracolpatria.com}}

\date{29 de Junio de 2018}

%%%%

\usepackage{Sweave}
\begin{document}
\Sconcordance{concordance:ProyFMerge.tex:ProyFMerge.Rnw:%
1 28 1 1 0 36 1}


\maketitle


\begin{abstract}
En este trabajo se muestra el **Indice de Desarrollo Humano** de Colombia por departamento. Donde a traves de estadística poblacional donde se mira el nivel de educación, nivel de salud y ingresos per capita de los distintos departamentos para poder determinar los departamentos más vulnerables y los que tienen un mejor indice. 
\end{abstract}

\section*{Introducción}

Aqui les presento mi investigacion sobre diversos indices sociales en el mundo. Los indices los conseguí de wikipedia, espero que les gusten mucho. Aqui les presento mi investigacion sobre diversos indices sociales en el mundo. Los indices los conseguí de wikipedia, espero que les gusten mucho.Aqui les presento mi investigacion sobre diversos indices sociales en el mundo. Los indices los conseguí de wikipedia, espero que les gusten mucho.Aqui les presento mi investigacion sobre diversos indices sociales en el mundo. Los indices los conseguí de wikipedia, espero que les gusten mucho.
Aqui les presento mi investigacion sobre diversos indices sociales en el mundo. Los indices los conseguí de wikipedia, espero que les gusten mucho.Aqui les presento mi investigacion sobre diversos indices sociales en el mundo. Los indices los conseguí de wikipedia, espero que les gusten mucho.Aqui les presento mi investigacion sobre diversos indices sociales en el mundo. Los indices los conseguí de wikipedia, espero que les gusten mucho.



Comencemos viendo que hay en la sección \ref{univariada} en la página \pageref{univariada}.

\clearpage


\input{paperVersion_7_univariada.tex}
\input{paperVersion_7_bivariada.tex}
\input{paperVersion_7_regresion.tex}


\bibliographystyle{apalike}
\renewcommand{\refname}{Bibliography}
\bibliography{Colombia}

\end{document}
