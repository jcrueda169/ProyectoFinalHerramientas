\documentclass{article}

\usepackage[utf8]{inputenc}
\usepackage{longtable}
\usepackage{authblk}
\usepackage{adjustbox}
\usepackage{natbib}
\usepackage{graphicx}
\graphicspath{ {Imagenes/} }
\usepackage{array}
\usepackage{tabu}
\usepackage{multirow}
\usepackage{amsmath}
\usepackage{adjustbox}


\title{Índices de Desarrollo: Correlación con la Población Colombiana}
% autores
\renewcommand\Authand{, y }
\author[1]{\normalsize Juan Sebastián Cortázar}
\author[2]{\normalsize María Alejandra Restrepo}
\author[3]{\normalsize Juan Camilo Rueda}


\affil[1,2,3]{\small  Universidad de los Andes\\
\texttt{{js.cortazar533,ma.restrepot,jc.rueda169}@uniandes.edu.col}}



\date{7 de Julio de 2018}

%%%%

\usepackage{Sweave}
\begin{document}
\Sconcordance{concordance:ProyFMerge.tex:ProyFMerge.Rnw:%
1 28 1 1 0 36 1}


\maketitle


\begin{abstract}

En este trabajo se muestra el **Índice de Desarrollo Humano** (IDH) de Colombia por departamento, donde a través de estadística poblacional se observa el nivel de educación, nivel de salud e ingresos per cápita de los distintos departamentos para poder determinar los departamentos más vulnerables y los que tienen un mejor índice. Adicional, se identifica una correlación entre la población y el IDH, por lo cual se hace una regresión entre el IDH y las poblaciones de cabecera por departamento y el total de población de cada departamento con el IDH.

\end{abstract}

\section*{Introducción}

El Índice de Desarrollo Humano es una medida utilizada para determinar el crecimiento y el desarrollo de las zonas y los países teniendo no solo en cuenta el crecimiento económico. Este índice busca medir no solo los PIB per cápita de las personas, sino que busca entender el acceso a salud y a la educación que determinan la posibilidad de crecimiento de las sociedades y sus capacidades de generar unas mejores condiciones de vida. 
El valor del IDH es la media geométrica entre los índices normalizados de las tres dimensiones (Salud, Educación y Nivel de Vida) como se muestra en la imagen \ref{IDH} a continuación. 


%imagen de calculo del IDH
\begin{figure}[h]
\centering
\includegraphics{hdiCalc}
\label {IDH}
\end{figure}

Para el cálculo de cada uno de los índices se tienen unos límites inferiores y superiores que en conjunto con los valores del sector (ya sea país o departamento) generan cada uno de los índices. En la tabla \ref{Tabla 1:} se puede ver lo siguiente.

%tabla boundary values

\begin{table}[h!]
\centering
  \begin{tabular}{l c c c}
  \hline
  Dimensiones & Indicador & Min & Max \\ [0.25ex]
  \hline \hline
  Salud & Expectancia de Vida (años) & 20 & 85 \\
  \multirow{2}{*}{Educación} & Escolaridad Adultos & 0 & 18 \\ 
   & Esperanza educativa niños & 0 & 15 \\
  Nivel de Vida  & PIB per Capita (USD ctes 2011) & 100 & 75,000 \\
  \hline
  \end{tabular}
 \caption {Rango de Dimensiones IDH}
  \label{Tabla 1:}
\end {table} 

La variable Salud se genera a través de un índice compuesto que refleja condiciones de salud en los hogares: protección de salud, a través del IGSS o de un seguro, número de personas por dormitorio, tipo de acceso a agua y saneamiento y tipo de piso en la vivienda. Todos estos factores influencian la expectantica de vida y se calculan de la siguiente manera.

\[ Salud=\frac{LE-20} {85-20} \]

Donde $LE = Expectativa de Vida$
La variable Educación es un indicador compuesto que incluye la escolaridad alcanzada por adultos mayores de 25 años y la esperanza educativa en niños. En el primer indicador se mide la tasa de alfabetización de adultos en el segundo se mide la tasa bruta combinada de matriculación en educación primara, secundario y superior, así como los años de duración de la educación obligatoria. El cálculo del índice de educación se define de la siguiente manera
\[Educación= \frac{EA + EN} {2} \]
Donde
\[EA= \frac{Prom de años de educación adultos} {18}  \]
\[EN= \frac{Prom de años de educación niños} {15}  \]

La variable del nivel de vida mide el PIB per cápita de una zona o país teniendo en cuenta un mínimo esperado y un maximo. La formula es la siguiente

\[Nivel de Vida = \frac {Ln(PIBx) – Ln (100)} {Ln(75,000)-Ln(100)} \]





Comencemos viendo que hay en la sección \ref{univariada} en la página \pageref{univariada}.

\clearpage


\section{Exploración Univariada}\label{univariada}


En esta sección nos interesa explorar cada indice (IDH), para esto se realizan varias estadisticas con la información obtenida. En primer lugar, se evalua el numero de datos y la mediana de cada uno de los tipos de población.En la tabla \ref{stats} en la página \pageref{stats}.





% Table created by stargazer v.5.2.2 by Marek Hlavac, Harvard University. E-mail: hlavac at fas.harvard.edu
% Date and time: sáb, jul 07, 2018 - 14:50:53
\begin{table}[!htbp] \centering 
  \caption{Medidas estadísticas} 
  \label{stats} 
\begin{tabular}{@{\extracolsep{5pt}}lcc} 
\\[-1.8ex]\hline 
\hline \\[-1.8ex] 
Statistic & \multicolumn{1}{c}{N} & \multicolumn{1}{c}{Median} \\ 
\hline \\[-1.8ex] 
cabecera & 32 & 717,197 \\ 
resto & 32 & 268,111.5 \\ 
\hline \\[-1.8ex] 
\end{tabular} 
\end{table} 
Para resaltar lo anterior, tenemos la Figura \ref{histograma} en la página \pageref{histograma}. 


%%%%% figure
\begin{figure}[h]
\centering
\includegraphics{Paper-histograma}
\caption{Histograma del IDH }
\label{histograma}
\end{figure}


Como las poblaciones tienen un sesgo se normalizan los datos con log, el histograma de estos nuevos datos se muestra en la Figura \ref{normal} en la página \pageref{normal}.

\begin{figure}[h]
\centering
\begin{adjustbox}{height=4cm}
\includegraphics{Paper-normal}
\end{adjustbox}
\caption{Histograma de poblaciones }
\label{normal}
\end{figure}


\section{Exploración Bivariada}\label{bivariada}


en esta sección nos interesa ver el impacto que tiene la población en el IDH, para esto se presenta en la tabla ref{corrDem} en la página \pageref{corrDem}. la correlación de las variables normalizadas con respecto al IDH
¿+}


% Table created by stargazer v.5.2.2 by Marek Hlavac, Harvard University. E-mail: hlavac at fas.harvard.edu
% Date and time: sáb, jul 07, 2018 - 14:50:53
\begin{table}[!htbp] \centering 
  \caption{Correlación de Democracia con las demás variables} 
  \label{corrDem} 
\begin{tabular}{@{\extracolsep{5pt}} cc} 
\\[-1.8ex]\hline 
\hline \\[-1.8ex] 
total & cabeLog \\ 
\hline \\[-1.8ex] 
$0.399$ & $0.487$ \\ 
\hline \\[-1.8ex] 
\end{tabular} 
\end{table} 


Ademas, se muestra la correlacion entre todas las variables independientes en la tabla \ref{corrTableX} en la página \pageref{corrTableX}

% Table created by stargazer v.5.2.2 by Marek Hlavac, Harvard University. E-mail: hlavac at fas.harvard.edu
% Date and time: sáb, jul 07, 2018 - 14:50:53
\begin{table}[!htbp] \centering 
  \caption{Correlación entre variables independientes} 
  \label{corrTableX} 
\begin{tabular}{@{\extracolsep{5pt}} ccc} 
\\[-1.8ex]\hline 
\hline \\[-1.8ex] 
 & total & cabeLog \\ 
\hline \\[-1.8ex] 
total & 1 &  \\ 
cabeLog & 0.71 & 1 \\ 
\hline \\[-1.8ex] 
\end{tabular} 
\end{table} 
los datos anteriores los puede ver visualmente en la figura  \ref{puntos} en la página \pageref{puntos}

\begin{figure}[h]
\centering
\includegraphics{Paper-puntos}
\caption{correlacion entre cabelog y restolog }
\label{puntos}
\end{figure}




\section{Modelos de Regresión}

Finalmente, vemos los modelos propuestos. En cada una se evalua la variable independiente DIH con cada una de las categorias de la poblacion. Los resultados se muestran en la Tabla \ref{regresiones} de la página \pageref{regresiones}.


% Table created by stargazer v.5.2.2 by Marek Hlavac, Harvard University. E-mail: hlavac at fas.harvard.edu
% Date and time: sáb, jul 07, 2018 - 14:50:53
\begin{table}[!htbp] \centering 
  \caption{Modelos de Regresión} 
  \label{regresiones} 
\begin{tabular}{@{\extracolsep{5pt}}lccc} 
\\[-1.8ex]\hline 
\hline \\[-1.8ex] 
 & \multicolumn{3}{c}{\textit{Dependent variable:}} \\ 
\cline{2-4} 
\\[-1.8ex] & \multicolumn{3}{c}{IDH} \\ 
\\[-1.8ex] & (1) & (2) & (3)\\ 
\hline \\[-1.8ex] 
 cabeLog & 0.013$^{***}$ &  &  \\ 
  & (0.004) &  &  \\ 
  & & & \\ 
 restoLog &  & 0.007 &  \\ 
  &  & (0.007) &  \\ 
  & & & \\ 
 totalLog &  &  & 0.013$^{**}$ \\ 
  &  &  & (0.005) \\ 
  & & & \\ 
 Constant & 0.634$^{***}$ & 0.722$^{***}$ & 0.629$^{***}$ \\ 
  & (0.055) & (0.082) & (0.068) \\ 
  & & & \\ 
\hline \\[-1.8ex] 
Observations & 32 & 32 & 32 \\ 
R$^{2}$ & 0.238 & 0.031 & 0.179 \\ 
Adjusted R$^{2}$ & 0.212 & $-$0.001 & 0.152 \\ 
Residual Std. Error (df = 30) & 0.037 & 0.042 & 0.039 \\ 
F Statistic (df = 1; 30) & 9.347$^{***}$ & 0.974 & 6.561$^{**}$ \\ 
\hline 
\hline \\[-1.8ex] 
\textit{Note:}  & \multicolumn{3}{r}{$^{*}$p$<$0.1; $^{**}$p$<$0.05; $^{***}$p$<$0.01} \\ 
\end{tabular} 
\end{table} 


















\endinput

\clearpage
%\documentclass{article}

%%%%
% PLOTS mapas
% eval=FALSE
% results=HIDE (verbatim default)
%%%%
%\usepackage[utf8]{inputenc}
%\usepackage{longtable}
%\usepackage{authblk}
%\usepackage{adjustbox}

%%%%
%\begin{document}
\Sconcordance{concordance:basico32departamentos.tex:basico32departamentos.Rnw:%
1 13 1 1 0 2 1 1 9 2 1 1 31 4 1 1 12 1 4 1 1 1 12 1 5 1 1 1 27 4 1 1 27 %
1 2 4 1}

%\SweaveOpts{concordance=TRUE}
%\SweaveOpts{concordance=TRUE}

%Partes extras para las nuevas columnas
% Exploracion Univariada --------------------------------------------------

\section{Exploración Espacial}

El siguiente mapa muestra el impacto que tiene la población de cada departamento sobre su respectivo IDH:





%con esto hagamos el merge:



 
 
\begin{figure}[h]
\centering
\begin{adjustbox}{width=8cm,height=6cm,clip,trim=1.5cm 2cm 0cm 2.5cm}
\includegraphics{basico32departamentos-plotMap0}
\end{adjustbox}
\caption{Impacto de población en IDH por departamento}\label{rawmap}
\end{figure}

Entre mayor es el impacto de la población sobre el IDH del departamento, más oscuro se muestra en el mapa. Se dividió el impacto en 3 niveles: BAJO, MEDIO Y ALTO.
Para lograr esta escala se implementó el siguiente procedimiento:

Primero se obtuvieron los datos de IDH, población de cabecera, el resto de la población y el total de la población de cada departamento del país.

Se limpiaron los datos reemplazando caracteres no reconocibles tales como la letra "ñ" y tildes.
Para evitar un sesgo significativo por variaciones amplias en número de habitantes, primero se utilizaron valores logarítmicos y luego se normalizaron. Finalmente se crearon las 3 agrupaciones por medio de la técnica K-means.


\endinput


%\end{document}




\bibliographystyle{apalike}
\renewcommand{\refname}{Bibliography}
\bibliography{Bibliografia}

\end{document}
